%% Copernicus Publications Manuscript Preparation Template for LaTeX Submissions
%% ---------------------------------
%% This template should be used for copernicus.cls
%% The class file and some style files are bundled in the Copernicus Latex Package, which can be downloaded from the different journal webpages.
%% For further assistance please contact Copernicus Publications at: production@copernicus.org
%% https://publications.copernicus.org/for_authors/manuscript_preparation.html


%% Please use the following documentclass and journal abbreviations for discussion papers and final revised papers.


%% 2-column papers and discussion papers
\documentclass[acp, manuscript]{copernicus}



%% Journal abbreviations (please use the same for discussion papers and final revised papers)

% Archives Animal Breeding (aab)
% Atmospheric Chemistry and Physics (acp)
% Advances in Geosciences (adgeo)
% Advances in Statistical Climatology, Meteorology and Oceanography (ascmo)
% Annales Geophysicae (angeo)
% ASTRA Proceedings (ap)
% Atmospheric Measurement Techniques (amt)
% Advances in Radio Science (ars)
% Advances in Science and Research (asr)
% Biogeosciences (bg)
% Climate of the Past (cp)
% Drinking Water Engineering and Science (dwes)
% Earth System Dynamics (esd)
% Earth Surface Dynamics (esurf)
% Earth System Science Data (essd)
% Fossil Record (fr)
% Geographica Helvetica (gh)
% Geoscientific Instrumentation, Methods and Data Systems (gi)
% Geoscientific Model Development (gmd)
% Geothermal Energy Science (gtes)
% Hydrology and Earth System Sciences (hess)
% History of Geo- and Space Sciences (hgss)
% Journal of Micropalaeontology (jm)
% Journal of Sensors and Sensor Systems (jsss)
% Mechanical Sciences (ms)
% Natural Hazards and Earth System Sciences (nhess)
% Nonlinear Processes in Geophysics (npg)
% Ocean Science (os)
% Proceedings of the International Association of Hydrological Sciences (piahs)
% Primate Biology (pb)
% Scientific Drilling (sd)
% SOIL (soil)
% Solid Earth (se)
% The Cryosphere (tc)
% Web Ecology (we)
% Wind Energy Science (wes)


%% \usepackage commands included in the copernicus.cls:
%\usepackage[german, english]{babel}
%\usepackage{tabularx}
%\usepackage{cancel}
%\usepackage{multirow}
%\usepackage{supertabular}
%\usepackage{algorithmic}
%\usepackage{algorithm}
%\usepackage{amsthm}
%\usepackage{float}
%\usepackage{subfig}
%\usepackage{rotating}


\begin{document}

\title{Emissions of trace gases from Australian temperate forest fires: emission factors and dependence on modified combustion efficiency}


% \Author[affil]{given_name}{surname}

\Author[1]{Elise-Andr\'ee}{Gu\'erette}
\Author[1]{Clare}{Paton-Walsh}
\Author[1]{Maximilien}{Desservettaz}
\Author[2]{Thomas E.L}{Smith}
\Author[3]{Liubov}{Volkova}
\Author[3]{Christopher J.}{Weston}
\Author[4]{C.P. (Mick)}{Meyer}

\affil[1]{Centre for Atmospheric Chemistry, School of Chemistry, University of Wollongong, Wollongong, NSW, Australia}
\affil[2]{Department of Geography, King's College London, London, UK}
\affil[3]{School of Ecosystem and Forest Sciences, the University of Melbourne, Creswick, VIC, Australia}
\affil[4]{CSIRO Oceans and Atmosphere Flagship, Aspendale, VIC, Australia}

%% The [] brackets identify the author with the corresponding affiliation. 1, 2, 3, etc. should be inserted.



\runningtitle{Emissions of trace gases from Australian temperate forest fires}

\runningauthor{Guerette et al.}

\correspondence{E-A. Gu\'erette (eag873@uowmail.edu.au)}



\received{}
\pubdiscuss{} %% only important for two-stage journals
\revised{}
\accepted{}
\published{}

%% These dates will be inserted by Copernicus Publications during the typesetting process.


\firstpage{1}

\maketitle



\begin{abstract}
We characterised trace gas emissions from Australian temperate forest fires through a mixture of in situ open-path FTIR measurements spectroscopy and selective ion flow tube mass spectrometry (SIFT-MS) and White cell FTIR spectroscopy of grab samples. We report emission factors for a total of 25 trace gas species measured in smoke from nine prescribed fires. We find significant dependence on modified combustion efficiency (MCE) for some species, although regional differences indicate that the use of MCE as a proxy may be limited. We also find that the fire-integrated MCE values derived from our in situ on-the-ground open-path measurements are not significantly different from those reported for airborne measurements of smoke from fires in the same ecosystem. We then compare our average emission factors to those measured for temperate forest fires elsewhere (North America) and for fires in another dominant Australian ecosystem (savanna) and find significant differences in both cases. Indeed, we find that although the emission factors of some species agree within 20$\%$, including those of hydrogen cyanide, ethene, methanol, formaldehyde and 1,3-butadiene; others, such as acetic acid, ethanol, monoterpenes, ammonia, acetonitrile and pyrrole, differ by a factor of two or more. This indicates that the use of ecosystem-specific emission factors is warranted for applications involving emissions from Australian forest fires.  
\end{abstract}


%\copyrightstatement{TEXT}

\end{document}